%% Version 4.3.2, 25 August 2014
%DIF LATEXDIFF DIFFERENCE FILE
%DIF DEL old_abstract.tex   Fri Feb 14 14:58:16 2020
%DIF ADD new_abstract.tex   Fri Feb 14 14:59:03 2020
%
%%%%%%%%%%%%%%%%%%%%%%%%%%%%%%%%%%%%%%%%%%%%%%%%%%%%%%%%%%%%%%%%%%%%%%
% Template.tex --  LaTeX-based template for submissions to the 
% American Meteorological Society
%
% Template developed by Amy Hendrickson, 2013, TeXnology Inc., 
% amyh@texnology.com, http://www.texnology.com
% following earlier work by Brian Papa, American Meteorological Society
%
% Email questions to latex@ametsoc.org.
%
%%%%%%%%%%%%%%%%%%%%%%%%%%%%%%%%%%%%%%%%%%%%%%%%%%%%%%%%%%%%%%%%%%%%%
% PREAMBLE
%%%%%%%%%%%%%%%%%%%%%%%%%%%%%%%%%%%%%%%%%%%%%%%%%%%%%%%%%%%%%%%%%%%%%

%% Start with one of the following:
% DOUBLE-SPACED VERSION FOR SUBMISSION TO THE AMS
\documentclass{ametsoc}

% TWO-COLUMN JOURNAL PAGE LAYOUT---FOR AUTHOR USE ONLY
% \documentclass[twocol]{ametsoc}

%%%%%%%%%%%%%%%%%%%%%%%%%%%%%%%%
%%% To be entered only if twocol option is used

\journal{waf}

%  Please choose a journal abbreviation to use above from the following list:
% 
%   jamc     (Journal of Applied Meteorology and Climatology)
%   jtech     (Journal of Atmospheric and Oceanic Technology)
%   jhm      (Journal of Hydrometeorology)
%   jpo     (Journal of Physical Oceanography)
%   jas      (Journal of Atmospheric Sciences)	
%   jcli      (Journal of Climate)
%   mwr      (Monthly Weather Review)
%   wcas      (Weather, Climate, and Society)
%   waf       (Weather and Forecasting)
%   bams (Bulletin of the American Meteorological Society)
%   ei    (Earth Interactions)

%%%%%%%%%%%%%%%%%%%%%%%%%%%%%%%%
%Citations should be of the form ``author year''  not ``author, year''
\bibpunct{(}{)}{;}{a}{}{,}

%%%%%%%%%%%%%%%%%%%%%%%%%%%%%%%%

%%% To be entered by author:

%% May use \\ to break lines in title:

\title{Verifying Operational Forecasts of Land-Sea Breeze and Boundary Layer Mixing Processes}

%%% Enter authors' names, as you see in this example:
%%% Use \correspondingauthor{} and \thanks{Current Affiliation:...}
%%% immediately following the appropriate author.
%%%
%%% Note that the \correspondingauthor{} command is NECESSARY.
%%% The \thanks{} commands are OPTIONAL.

    %\authors{Author One\correspondingauthor{Author One, 
    % American Meteorological Society, 
    % 45 Beacon St., Boston, MA 02108.}
% and Author Two\thanks{Current affiliation: American Meteorological Society, 
    % 45 Beacon St., Boston, MA 02108.}}

\authors{Ewan Short\correspondingauthor{School of Earth Sciences, The University of Melbourne, Melbourne, Victoria, Australia.}} 

\email{shorte1@student.unimelb.edu.au}

%% Follow this form:
    % \affiliation{American Meteorological Society, 
    % Boston, Massachusetts.}

\affiliation{School of Earth Sciences, and ARC Centre of Excellence for Climate Extremes, The University of Melbourne, Melbourne, Victoria, Australia.}

%% Follow this form:
    %\email{latex@ametsoc.org}

%\email{}

%% If appropriate, add additional authors, different affiliations:
    %\extraauthor{Extra Author}
    %\extraaffil{Affiliation, City, State/Province, Country}

\DeclareMathOperator{\mse}{mse} 
\DeclareMathOperator{\cov}{cov} 
\DeclareMathOperator{\var}{var} 
\DeclareMathOperator{\pr}{Pr} 

%\extraauthor{}
%\extraaffil{}

%% May repeat for a additional authors/affiliations:

%\extraauthor{}
%\extraaffil{}

%%%%%%%%%%%%%%%%%%%%%%%%%%%%%%%%%%%%%%%%%%%%%%%%%%%%%%%%%%%%%%%%%%%%%
% ABSTRACT
%
% Enter your abstract here
% Abstracts should not exceed 250 words in length!
%
% For BAMS authors only: If your article requires a Capsule Summary, please place the capsule text at the end of your abstract
% and identify it as the capsule. Example: This is the end of the abstract. (Capsule Summary) This is the capsule summary. 

% Run "latexdiff --append-context2cmd="abstract" short18_diurnal_cycles_winds.tex short18_diurnal_cycles_winds_revised.tex > short18_diurnal_cycles_winds_tracked_changes.tex" to track changes



\usepackage{comment}
%DIF PREAMBLE EXTENSION ADDED BY LATEXDIFF
%DIF UNDERLINE PREAMBLE %DIF PREAMBLE
\RequirePackage[normalem]{ulem} %DIF PREAMBLE
\RequirePackage{color}\definecolor{RED}{rgb}{1,0,0}\definecolor{BLUE}{rgb}{0,0,1} %DIF PREAMBLE
\providecommand{\DIFadd}[1]{{\protect\color{blue}\uwave{#1}}} %DIF PREAMBLE
\providecommand{\DIFdel}[1]{{\protect\color{red}\sout{#1}}}                      %DIF PREAMBLE
%DIF SAFE PREAMBLE %DIF PREAMBLE
\providecommand{\DIFaddbegin}{} %DIF PREAMBLE
\providecommand{\DIFaddend}{} %DIF PREAMBLE
\providecommand{\DIFdelbegin}{} %DIF PREAMBLE
\providecommand{\DIFdelend}{} %DIF PREAMBLE
%DIF FLOATSAFE PREAMBLE %DIF PREAMBLE
\providecommand{\DIFaddFL}[1]{\DIFadd{#1}} %DIF PREAMBLE
\providecommand{\DIFdelFL}[1]{\DIFdel{#1}} %DIF PREAMBLE
\providecommand{\DIFaddbeginFL}{} %DIF PREAMBLE
\providecommand{\DIFaddendFL}{} %DIF PREAMBLE
\providecommand{\DIFdelbeginFL}{} %DIF PREAMBLE
\providecommand{\DIFdelendFL}{} %DIF PREAMBLE
%DIF END PREAMBLE EXTENSION ADDED BY LATEXDIFF

\begin{document}

\abstract{}

\maketitle

\section{Introduction}
\label{Sec:Introduction}
\DIFdelbegin \DIFdel{Forecasts issued by the Australian }\DIFdelend \DIFaddbegin \DIFadd{Forecasters working for Australia's }\DIFaddend Bureau of Meteorology (BoM) \DIFdelbegin \DIFdel{are based on automated post-processed model data that is edited by human forecasters}\DIFdelend \DIFaddbegin \DIFadd{produce a seven day forecast in two key steps: first they choose a model guidance dataset to base the forecast on, then they use graphical software to manually edit this data}\DIFaddend . Two types of edits are commonly made to the wind fields \DIFdelbegin \DIFdel{. These edits }\DIFdelend \DIFaddbegin \DIFadd{that }\DIFaddend aim to improve how the influences of boundary layer mixing and land-sea breeze processes are represented in the forecast. In this study \DIFdelbegin \DIFdel{we }\DIFdelend \DIFaddbegin \DIFadd{I }\DIFaddend compare the diurnally varying component of the BoM's official \DIFdelbegin \DIFdel{edited }\DIFdelend wind forecast, with that of station observations and unedited model \DIFdelbegin \DIFdel{datasets, to assess changes to error and bias resulting from these edits. We }\DIFdelend \DIFaddbegin \DIFadd{guidance datasets. I }\DIFaddend consider coastal locations across Australia over June, July and August 2018, aggregating data over three spatial scales. The edited forecast generally only produces a lower mean absolute error than model guidance at the coarsest spatial scale (over fifty thousand square kilometres), but can achieve lower seasonal biases over all spatial scales. However, the edited forecast only reduces errors or biases at particular times and locations, and rarely produces lower errors or biases than all model guidance products simultaneously. To better understand \DIFdelbegin \DIFdel{the }\DIFdelend \DIFaddbegin \DIFadd{physical reasons for }\DIFaddend biases in the \DIFaddbegin \DIFadd{mean }\DIFaddend diurnal wind cycles, \DIFdelbegin \DIFdel{we }\DIFdelend \DIFaddbegin \DIFadd{I }\DIFaddend fit modified ellipses to the \DIFdelbegin \DIFdel{temporal hodographs of }\DIFdelend seasonally averaged diurnal wind \DIFdelbegin \DIFdel{cycles}\DIFdelend \DIFaddbegin \DIFadd{temporal hodographs}\DIFaddend . Biases in the official forecast diurnal cycle vary with location for multiple reasons, including biases in the directions sea-breezes approach coastlines, amplitude  \DIFdelbegin \DIFdel{and shape biasesin the hodographs}\DIFdelend \DIFaddbegin \DIFadd{biases}\DIFaddend , and disagreement \DIFdelbegin \DIFdel{as to whether }\DIFdelend \DIFaddbegin \DIFadd{in the relative contribution of }\DIFaddend sea-breeze \DIFdelbegin \DIFdel{or }\DIFdelend \DIFaddbegin \DIFadd{and }\DIFaddend boundary layer mixing processes \DIFdelbegin \DIFdel{contribute most to the }\DIFdelend \DIFaddbegin \DIFadd{to the mean }\DIFaddend diurnal cycle.

\bibliographystyle{ametsoc2014}
\bibliography{./references.bib}

\end{document}
