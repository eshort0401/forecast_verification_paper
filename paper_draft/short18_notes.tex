%%%%%%%%%%%%%%%%%%%%%%%%%%%%%%%%%%%%%%%%%%%%%%%%%%%%%%%
% A template for Wiley article submissions.
% Developed by Overleaf. 
%
% Please note that whilst this template provides a 
% preview of the typeset manuscript for submission, it 
% will not necessarily be the final publication layout.
%
% Usage notes:
% The "blind" option will make anonymous all author, affiliation, correspondence and funding information.
% Use "num-refs" option for numerical citation and references style.
% Use "alpha-refs" option for author-year citation and references style.

\documentclass[alpha-refs]{wiley-article}
% \documentclass[blind,num-refs]{wiley-article}

% Add additional packages here if required
\usepackage{siunitx}

% Update article type if known
\papertype{Original Article}
% Include section in journal if known, otherwise delete
\paperfield{Journal Section}

\title{Land-Sea Breeze Forecast Verification}

% Include full author names and degrees, when required by the journal.
% Use the \authfn to add symbols for additional footnotes and present addresses, if any. Usually start with 1 for notes about author contributions; then continuing with 2 etc if any author has a different present address.
\author[1]{Ewan Short}
\author[2]{Ben Price}
\author[3]{Derryn Griffiths}
\author[3]{Michael Foley}
%\author[2\authfn{2}]{Author Three PhD}
%\author[2]{Author B.~Four}

%\contrib[\authfn{1}]{Equally contributing authors.}

% Include full affiliation details for all authors
\affil[1]{ARC Centre of Excellence for Climate Extremes, School of Earth Sciences, University of Melbourne, Parkville, VIC, 3010, Australia}
\affil[2]{Bureau of Meteorology, Casuarina, NT, 0810, Australia}
\affil[3]{Bureau of Meteorology, Melbourne, VIC, 3208, Australia}

\corraddress{Ewan Short, ARC Centre of Excellence for Climate Extremes, School of Earth Sciences, University of Melbourne, Parkville, VIC, 3010, Australia}
\corremail{ewan.short@unimelb.edu.au}

%\presentadd[\authfn{2}]{Department, Institution, City, State or Province, Postal Code, Country}

\fundinginfo{ARC Centre of Excellence for Climate System Science}

% Include the name of the author that should appear in the running header
\runningauthor{Ewan Short et al.}

\begin{document}

\maketitle

\noindent Although they have similar definitions, $\overline{\text{WPI}}$ and CWPI measure different things. They do not converge as the length of the time period grows - they don't even necessarily approach the same sign. As a simple example, suppose that for each day, the observed and Official wind perturbations are given by $\boldsymbol{p}_{\text{AWS}} = \left(5\cos\omega t , 5\sin\omega t\right)$ and $\boldsymbol{p}_\text{O} = \left(6\cos\omega t , 6\sin\omega t\right)$, respectively. Furthermore, suppose that the ACCESS perturbations alternate between $\boldsymbol{p}_{\text{A}} = \left(7\cos\omega t , 7\sin\omega t\right)$ and $\boldsymbol{p}_{\text{A}} = \left(3\cos\omega t , 3\sin\omega t\right)$ from one day to the next. Then for any contiguous period of $n$ days, $\overline{\text{WPI}} = 2 - 1 = 1$, but $\text{CWPI} \approx -1$, with the approximation becoming exact for even $n$. Moreover $\overline{\text{WPI}}=1$ with a confidence of 1, and using the bootstrapping procesure described above, the confidence that $\text{CWPI} = -1$ approaches 1 as $n\to \infty$. 

This example shows that while the WPI and CWPI are sensitive both to random error and consistent biases between the different datasets, the CWPI becomes increasingly less sensitive to random error as the length of the time period being considered grows. Thus while the WPI arguably provides a more meaningful operational metric, as it measures the accuracy of actual forecast data, it may favour a more biased dataset over a less biased one, just because the internal variability of that dataset is lower. One consequence of this is that model data at a lower spatiotemporal resolution may outperform in $\overline{\text{WPI}}$ model data of a higher resolution, purely because the internal variability is lower. In this way, the CWPI may actually provide more information about the performance of different forecasts.

\bibliography{./Coastal_Winds.bib}

\end{document}
