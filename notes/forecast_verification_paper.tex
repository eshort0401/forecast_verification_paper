\documentclass[12pt]{article}

%\usepackage[T1]{fontenc}
% Nicer default font (+ math font) than Computer Modern for most use cases
% \usepackage{mathpazo}

% Basic figure setup, for now with no caption control since it's done
% automatically by Pandoc (which extracts ![](path) syntax from Markdown).
\usepackage{graphicx}
\usepackage{natbib}
\usepackage{har2nat}
\usepackage{xcolor} % Allow colors to be defined
\usepackage{enumerate} % Needed for markdown enumerations to work
\usepackage{geometry} % Used to adjust the document margins
\usepackage{amsmath} % Equations
\usepackage{amssymb} % Equations
\usepackage[mathletters]{ucs} % Extended unicode (utf-8) support
\usepackage[utf8x]{inputenc} % Allow utf-8 characters in the tex document
\usepackage{fancyvrb} % verbatim replacement that allows latex
\usepackage{grffile} % extends the file name processing of package graphics
                     % to support a larger range
% The hyperref package gives us a pdf with properly built
% internal navigation ('pdf bookmarks' for the table of contents,
% internal cross-reference links, web links for URLs, etc.)
\usepackage{hyperref}
\usepackage{setspace}
\usepackage[font={small, stretch=1.3}, labelfont=bf]{caption}
\renewcommand{\floatpagefraction}{.7}
\usepackage{longtable} % longtable support required by pandoc >1.10
\usepackage{booktabs}  % table support for pandoc > 1.12.2
\usepackage[inline]{enumitem} % IRkernel/repr support (it uses the enumerate* environment)
\usepackage[nottoc,numbib]{tocbibind}

% Colors for the hyperref package
\definecolor{urlcolor}{rgb}{0,.145,.698}
\definecolor{linkcolor}{rgb}{0,0,0}
\definecolor{citecolor}{rgb}{0,0,0}

\DeclareMathOperator{\sgn}{sgn}

\sloppy
% Setup hyperref package
\hypersetup{
    breaklinks=true,  % so long urls are correctly broken across lines
    colorlinks=true,
    urlcolor=urlcolor,
    linkcolor=linkcolor,
    citecolor=citecolor,
}

\geometry{verbose,tmargin=2cm,bmargin=2cm,lmargin=2.5cm,rmargin=2cm}
\linespread{1.3}

\title{Verification of Land-Sea Breeze Forecasts}
\author{Ewan Short}
\date{\today}

\begin{document}

\maketitle

\begin{abstract}
\noindent This report presents a methodology for comparing the performance of official Bureau forecasts of daily wind processes - such as the land-sea breeze - to unedited model guidance products like that of the European Center for Medium-Range Weather Forecasting (ECMWF). The methodology is then applied to a case study of the Darwin airport station during the dry season months of June, July and August 2017, and the wet season months of December, January and February 2017/18. The results indicate that during the dry season the ECMWF sea-breezes are generally more accurate than those of the official forecast. However, during the wet season, this result is reversed, and the official forecast sea-breezes generally outperform those of ECMWF. In both seasons boundary layer mixing processes are generally represented better in official forecasts than in ECMWF.  
\end{abstract}

\clearpage

\section{Introduction}\label{introduction}
This report describes a methodology for assessing how well the Australian Bureau of Meteorology's official forecast products capture diurnal (i.e daily) wind processes compared to model guidance products like the Operational Consensus Forecast (OCF), the Australian Community Climate and Earth-System Simulator (ACCESS) model, and the European Center for Medium-Range Weather Forecasting (ECMWF) model. The Bureau's official forecast is typically constructed from a blend of model outputs that are then edited by human forecasters, whereas OCF, ACCESS and ECMWF represent unedited model output. Comparing these products therefore provides insight into the added value provided by human forecasters. The analysis presented here is based purely on accuracy in a physical science sense; the analysis cannot account for the social, political or economic reasons human forecasters may have for over or under forecasting certain events or processes.

The analysis provided in this report was performed with software tools I developed using the Bureau's Jive database and code libraries with the assistance of Ben Price and Nicholas Loveday; the code can be found on the Bureau's Jupyterhub server.\footnote{Contact Ben Price (ben.price@bom.gov.au) or Nicholas Loveday (nicholas.loveday@bom.gov.au) for more information.} The code is completely general - any location and time period can be analysed - but to provide a concrete example this report will present a case study of the Darwin airport station during the dry season months of June, July, August of 2017, and the wet season months of December, January, February of 2017/18. 

\section{Methods} \label{methods}
The problem of assessing how well the Bureau's official forecast captures diurnal wind processes has been approached in the most general way possible. Although close to coastlines the land-sea breeze is generally the dominant diurnal wind process, the overall diurnal signal may include mountain-valley breezes, boundary layer mixing processes, atmospheric tides, and urban heat island circulations. Forecasters typically edit model output to account for \emph{both} unresolved sea-breezes \emph{and} unresolved boundary layer mixing; attempting to focus solely on sea-breezes without examining the entire diurnal cycle may thus risk mistaken conclusions, with the effect of one process mistaken for another.

Diurnal wind signals are therefore analysed for each hour of the day. This is done by taking hourly automatic weather station, official forecast and model wind data and subtracting a twenty hour centred running mean from each data point; this provides a collection of wind \emph{perturbation} datasets.\footnote{Unless otherwise stated all vector operations have their usual cartesian-coordinate, componentwise definitions.} 

Thinking of land-sea breezes in terms of perturbations may require a conceptual shift from the usual operational definitions. A forecaster would likely define a sea-breeze to be a reversal in wind direction from a primarily offshore flow during the night and morning, to an onshore flow in the afternoon and evening. However, even if the wind is offshore the entire day, sea-breeze \emph{perturbations} are generally still detectable as a weakening of the offshore flow throughout the afternoon and evening. 

The methods developed in this report can be readily extended to analyse \emph{just} the sea-breezes satisfying the operational definition above. For instance, to study the sea-breezes at a station near a coastline with inward pointing normal vector $\widehat{\boldsymbol{n}}$, the wind perturbation datasets could be restricted to just those days where the corresponding raw wind vector $\boldsymbol{u}$ satisfies 
\begin{equation}
\widehat{\boldsymbol{n}} \cdot \boldsymbol{u} > 0
\end{equation}
for at least one of the hours of that day.\footnote{This and similar techniques could easily be implemented in the existing Jupyter notebook using boolean masking.}

Two closely related indices for assessing the performance of official forecast wind perturbations against those of an unedited model guidance product have been developed; these are outlined below. 

\subsection{The Wind Perturbation Index}\label{daily-performance}
The methodology for comparing the official forecast to one of the model
guidance products (OCF, ACCESS or ECMWF) is based on comparing each to station
observations, and can be summarised as follows. 
\begin{enumerate}
\item
Isolate the diurnal signal in the observational data, the official forecast and model guidance by subtracting a twenty four hour centred running mean at each time step. The resulting datasets describe the wind \emph{perturbations} from the 24 hour average \emph{background wind}. 
\item
At each time step calculate the quantity
\begin{equation}
\text{WPI}_{\text{off}} \equiv \left\lvert \boldsymbol{u}_{\text{obs}}-\boldsymbol{u}_{\text{off}} \right\rvert 
\end{equation}
which is the magnitude of the vector difference between the observed and
official forecast wind perturbations. Calculate
\(\text{WPI}_{\text{mod}}\) analogously, and define the \emph{Wind
Perturbation Index}
\begin{equation}
\text{WPI} \equiv \text{WPI}_{\text{mod}} - \text{WPI}_{\text{off}}.
\end{equation}
The official forecast wind perturbation at a given time will be closer
to the observed perturbation than that of the model guidance if and only if \(\text{WPI} > 0\). 
\item
Aggregate \(\text{WPI}\) on an hourly basis, and calculate the mean, $\overline{\text{WPI}}$, and the sampling distribution of the mean, for each hour. Use the sampling distribution to work out the likelihood that $\overline{\text{WPI}} > 0$.    
\end{enumerate}

Note the first step is necessary because if raw wind fields are compared, differences in winds between datasets will be dominated by the typically larger differences in the daily mean winds of each dataset.
Put another way, the \emph{changes} in the wind throughout a given day in a given forecast may match the way the winds change in observations, \emph{even if} the observed and forecast winds are different. Therefore, not subtracting a daily average background wind may result in the conclusion that diurnal processes (like land-sea breezes) are being treated poorly, when actually the errors come from processes occurring on longer timescales, such as poorly forecast synoptic pressure systems or monsoon processes. 

\subsection{The Climatological Wind Perturbation Index}\label{climatological-performance}
Although the above methodology is perhaps the most relevant for assessing forecast performance in an operational sense, it is also informative to think about how well each forecast product performs in a
climatological sense, i.e to ask how well the \emph{mean} forecast perturbation winds match the \emph{mean} observed perturbations over a suitable climatological period. One reason for doing this is that the diurnal signal becomes much clearer when perturbations are averaged over a number of days and random variability is smoothed out. If the goal is to assess how forecasts and models capture \emph{regular} diurnal wind processes like land-sea breezes that occur at roughly the same times each day, then comparing perturbation climatologies is arguably a better option: comparing perturbations on a day to day basis will also implicitly assess how different datasets resolve \emph{irregular} processes at daily and shorter timescales; for instance turbulence and cold pool dynamics.

To assess performance on a climatological basis, steps 2 and 3 above are modified as follows. 
\begin{enumerate}
\setcounter{enumi}{1}
\item
Average the perturbations at each hour across the climatological period, i.e average all the 00:00 UTC perturbations, all the 01:00 UTC perturbations, and so forth. Calculate the quantity
\begin{equation}
\text{CWPI}_{\text{off}} \equiv \left\lvert \overline{\boldsymbol{u}}_{\text{obs}}-\overline{\boldsymbol{u}}_{\text{off}} \right\rvert.
\end{equation}
This represents the magnitude of the vector difference between the \emph{mean} 
observed wind perturbations and \emph{mean} official forecast wind perturbations. Calculate \(\text{CWPI}_{\text{mod}}\) analogously and define the the \emph{Climatological Wind Perturbation Index}
\begin{equation}
\text{CWPI} = \text{CWPI}_{\text{mod}} - \text{CWPI}_{\text{off}}.
\end{equation}
\item
Estimate the sampling distribution of \(\text{CWPI}\) by bootstrapping
\citep{efron79}. Use the sampling distribution to calculate the likelihood that $\text{CWPI} > 0$.  
\end{enumerate}

\section{Conclusion}
In this report, a methodology for comparing the performance of Bureau forecasts of diurnal wind processes to unedited model guidance products has been developed and applied to a case study of the Darwin airport. The key results may be summarised as follows.
\begin{enumerate}
\item
During the dry season months of June, July and August 2017, the ECMWF sea-breeze is generally more accurate than that of the official forecast. However, during the wet season months of December, January and February 2017/18 this result is reversed, and the official forecast sea-breeze generally outperforms that of ECMWF. 
\item
In both seasons, boundary layer mixing processes are generally represented better in official forecasts than in ECMWF.
\item
In the dry season, the climatological wind perturbations of the official forecast generally outperform those of ECMWF between 13:00 and 16:00 UTC. This is due to ECMWF not capturing the magnitude of the south-easterly mean perturbations. 
\item
During the wet season, the climatological wind perturbations of the official forecast generally outperform those of ECMWF at 11:00 UTC. This is due to ECMWF underestimating the magnitude of the mean land-breeze perturbation.
\end{enumerate}

There a number of ways that this work could be extended. The most pressing would probably be to investigate whether the results presented here change when a more operational definition of the sea breeze is used in place of the entirely perturbation based definition used here: this could be done using the method described in section \ref{methods}.

Following this, a nationwide study could be conducted focusing on the most operationally relevant locations of each state, for instance, airport stations. This should be done on a seasonal basis given that the examples considered here indicate results are seasonally dependent. The boundary layer mixing and sea-breeze editing techniques used by forecasters could then be collated and compared, with a view to standardising them across the country and optimising performance.

\bibliographystyle{agsm}
\bibliography{./Coastal_Winds.bib}

\end{document}
